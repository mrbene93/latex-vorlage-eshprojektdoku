%!TEX encoding = UTF-8
%% Präambel
\documentclass[a4paper,12pt]{article}

% Pakete
\usepackage[utf8]{inputenc}
\usepackage[ngerman]{babel}
\usepackage[ngerman]{translator}
\usepackage[T1]{fontenc}
\usepackage{fontspec}
\usepackage{geometry}
\usepackage{fancyhdr}
\usepackage{graphicx}
\usepackage[font={small,it}]{caption}
\usepackage{color}
\usepackage{hyperref}
\usepackage[toc, nopostdot]{glossaries}
\usepackage{newclude}
\usepackage{tikz}
\usepackage{lscape}
\usepackage{makeidx}
\usepackage{ragged2e}
\usepackage{verbatim}
\usepackage{listings}
\usepackage{eurosym}
\usepackage{dirtree}
\usepackage{pgfgantt}

% Variablen
\newcommand{\autorname}{Peter Beispiel}
\newcommand{\dokumententitel}{Hier steht der beispielhafte Dokumententitel}
\newcommand{\ausbildungsberuf}{Fachinformatiker FR Beispiel}
\newcommand{\projektzeitraum}{01.01.0000 -- 02.02.0000}
\newcommand{\ptidentnummer}{123456}
\newcommand{\ptadresse}{Beispielstraße 10, 12345 Beispiel}
\newcommand{\pttelefon}{0000 12 34567}
\newcommand{\ptemail}{peter.beispiel@esh.essen.de}
\newcommand{\pbname}{Klaus Knüppel}
\newcommand{\pbadresse}{Beispielstraße 20, 54321 Beispiel}
\newcommand{\pbtelefon}{0000 98 7654}
\newcommand{\pbemail}{klaus.knueppel@esh.essen.de}

% Schriftart
\setmainfont[
    ItalicFont = Source Sans Pro Light Italic,
    BoldFont = Source Sans Pro Semibold,
    BoldItalicFont = Source Sans Pro Semibold Italic
]{Source Sans Pro Light}
\setmonofont[
    ItalicFont = Source Code Pro Light Italic,
    BoldFont = Source Code Pro Semibold,
    BoldItalicFont = Source Code Pro Semibold Italic
]{Source Code Pro Light}

% Farben
% \definecolor{weiß}{HTML}{FFFFFF}

% Formatierung
\sloppy
\geometry{a4paper, top=28mm, bottom=20mm, left=22mm, right=22mm}
\hypersetup{colorlinks, citecolor=black, filecolor=black, linkcolor=black, urlcolor=black}
\fancyhf{}
\fancyhead[L]{\begin{picture}(0,0) \put(4,-2){\includegraphics[width=40mm]{bilder/esh-logo-flach.png}} \end{picture}}
\fancyhead[R]{\nouppercase{\leftmark}}
\fancyfoot[L]{\autorname}
\fancyfoot[C]{Projektdokumentation}
\fancyfoot[R]{\thepage}

% Sonstiges
\setlength\parindent{0pt}
\makeindex

% Glossar
%!TEX encoding = UTF-8

% \newglossaryentry{Beispiel}{name={Beispiel}, description={Hier steht eine beispielhafte Beschreibung.}}

\makeglossaries{}

% Daten
\author{\autorname}
\title{\dokumententitel}


%% Inhalt
\begin{document}
    \pagenumbering{arabic}
    %!TEX encoding = UTF-8
\begin{titlepage}
    \centering\ \\
    \vspace{10mm}
    \begin{figure}[h]
        \centering
        \includegraphics[width=0.6\textwidth]{bilder/esh-logo.png}
    \end{figure}
    \vspace{15mm}
    {\LARGE\textbf{\dokumententitel}}\\
    \vspace{25mm}
    {\Large\textbf{Dokumentation zur betrieblichen Projektarbeit}}\\
    \vspace{10mm}
    {\large Projektbericht für die IHK Abschlussprüfung zum\\\ausbildungsberuf\\\ \\
        Projektzeitraum \projektzeitraum}\\
    \vfill
    \begin{minipage}{0.49\textwidth}
        \begin{flushleft}
            \textbf{Prüfungsteilnehmer:}\\
            \autorname\\
            Identnummer: \ptidentnummer\\
            \ptadresse\\
            Telefon: \pttelefon\\
            E-Mail: \ptemail
        \end{flushleft}
    \end{minipage}
    \begin{minipage}{0.49\textwidth}
        \begin{flushright}
            \textbf{Ausbildungsbetrieb:}\\
            Essener Systemhaus\\
            Betrieb der Stadt Essen\\
            Kruppstraße 82 -- 100, 45145 Essen\\
            \textbf{Projektbetreuer:}\\
            \pbname\\
            Telefon: \pbtelefon\\
            E-Mail: \pbemail
        \end{flushright}
    \end{minipage}
\end{titlepage}

    %!TEX encoding = UTF-8
\thispagestyle{empty}
\tableofcontents

    \pagestyle{fancy}
    \setcounter{page}{1}
    % Weitere includes

    \appendix
    \pagenumbering{Roman}
    \fancyhead{}
    \fancyfoot[L]{}
    \fancyfoot[C]{}
    \setcounter{page}{1}
    \printglossary[style=altlist, title=Glossar, toctitle=Glossar, numberedsection, nonumberlist]
\end{document}
